\documentclass[11pt]{report} 
\usepackage[utf8]{inputenc} % set input encoding (not needed with XeLaTeX)

\usepackage{amsmath}
%%% PAGE DIMENSIONS
\usepackage{geometry} 
\geometry{a4paper} 
% \geometry{margin=2in} % for example, change the margins to 2 inches all round
% \geometry{landscape} % set up the page for landscape

\usepackage{graphicx} 
\usepackage{array}
% \usepackage[parfill]{parskip} % Activate to begin paragraphs with an empty line rather than an indent


\title{Compréhension et analyse des systèmes de fichiers}
\author{BELLEC Vincent \and PetroleVB}

\begin{document}
	
	\maketitle
	\tableofcontents

\chapter{FAT}

\section{Boot Sector}

Le secteur de boot est  un secteur (donc de 512 octets) qui va contenir des informations essentielles sur les fat et qui fini normalement par 55 AA.\\
Les Nombres sont tous écris en BIG Endian dans la mémoire, ainsi, 0x0002 par exemple doit s'inverser en 0x0200, soit 512.\\
\subsection{FAT12/FAT16}

\begin{verbatim}
	00000000 | eb 3c 90 6d 6b 64 6f 73 | 66 73 00 00 02 01 01 00   |.<.mkdosfs......| 
	00000010 | 02 e0 00 40 0b f0 09 00 | 12 00 02 00 00 00 00 00   |...@............| 
	00000020 | 00 00 00 00 00 00 29 a5 | 0e ae eb 20 20 20 20 20   |......)....     | 
	00000030 | 20 20 20 20 20 20 46 41 | 54 31 32 20 20 20 0e 1f   |      FAT12   ..| 
	00000040 | be 5b 7c ac 22 c0 74 0b | 56 b4 0e bb 07 00 cd 10   |.[|.".t.V.......| 
	00000050 | 5e eb f0 32 e4 cd 16 cd | 19 eb fe 54 68 69 73 20   |^..2.......This | 
	00000060 | 69 73 20 6e 6f 74 20 61 | 20 62 6f 6f 74 61 62 6c   |is not a bootabl| 
	00000070 | 65 20 64 69 73 6b 2e 20 | 20 50 6c 65 61 73 65 20   |e disk.  Please |
	00000080 | 69 6e 73 65 72 74 20 61 | 20 62 6f 6f 74 61 62 6c   |insert a bootabl| 
	00000090 | 65 20 66 6c 6f 70 70 79 | 20 61 6e 64 0d 0a 70 72   |e floppy and..pr| 
	000000a0 | 65 73 73 20 61 6e 79 20 | 6b 65 79 20 74 6f 20 74   |ess any key to t| 
	000000b0 | 72 79 20 61 67 61 69 6e | 20 2e 2e 2e 20 0d 0a 00   |ry again ... ...| 
	000000c0 | 00 00 00 00 00 00 00 00 | 00 00 00 00 00 00 00 00   |................| 
	*
	000001f0 | 00 00 00 00 00 00 00 00 | 00 00 00 00 00 00 55 aa   |..............U.| 
\end{verbatim}
\newpage

16 premiers octets : \\
3 Octets : eb 3c 90 : Saut vers un programme qui va charger le système d'exploitation \\
8 Octets : mkdosfs : Nom du programme qui a formaté le disque\\
2 Octets : 00 02 : Nombre d'octets par secteur (512, 1 024, 2 048 ou 4 096). \\
1 Octet : 01 : Nombre de secteurs par cluster (1, 2, 4, 8, 16, 32, 64 ou 128).\\
2 Octets : 01 00 : Nombre de secteurs réservés en comptant le secteur de boot\\
(32 par défaut pour FAT32, 1 par défaut pour FAT12/16).
\begin{verbatim}
00000000 | eb 3c 90 6d 6b 64 6f 73 | 66 73 00 00 02 01 01 00   |.<.mkdosfs......| 
\end{verbatim}

16 seconds octets :\\
1 Octet : 02 : Nombre de FATs sur le disque\\
2 Octets : e0 00 : Taille du répertoire racine en nombre d'entrées.\\
2 Octets : 40 0b : Nombre total de secteurs 16-bit.\\
1 Octet : f0 : Type de disque \\
F0     2,88 Mo    3,5 pouces, 2 faces, 36 secteurs\\
F0     1,44 Mo    3,5 pouces, 2 faces, 18 secteurs\\
F9     720 Ko     3,5 pouces, 2 faces, 9 secteurs\\
F9     1,2 Mo     5,25 pouces, 2 faces, 15 secteurs\\
FD     360 Ko     5,25 pouces, 2 faces, 9 secteurs\\
FF     320 Ko     5,25 pouces, 2 faces, 8 secteurs\\
FC     180 Ko     5,25 pouces, 1 face, 9 secteurs\\
FE     160 Ko     5,25 pouces, 1 face, 8 secteurs\\
F8     -----      Disque dur\\
2 Octets : 09 00 : Taille d'une FAT en secteurs.\\
2 Octets : 12 00 : Nombre de secteurs par piste.\\
2 Octets : 02 00 : Nombre de têtes.\\
4 Octets : 00 00 00 00 : Secteurs cachés (0 par défaut si le disque n'est pas partitionné).
\begin{verbatim}
	00000010 | 02 e0 00 40 0b f0 09 00 | 12 00 02 00 00 00 00 00   |...@............| 
\end{verbatim}

Spécifiques aux FAT12 - FAT16 : \\
4 Octets : 00 00 00 00 : Nombre total de secteurs 32-bit.\\
1 Octet : 00 : Identifiant du disque (à partir de 0x00 pour les disques amovibles et à partir de 0x80 pour les disques fixes).\\
1 Octet : 00 : Réservé pour usage ultérieur.\\
1 Octet : 29 : Signature (0x29 par défaut).\\
4 Octets : a5 0e ae eb : Numéro de série du disque.\\
11 Octets : 20 20 20 20 20 20 20 20 20 20 20 : Nom du disque sur 11 caractères.\\
8 Octets : 46 41 54 31 32 20 20 20 : Type de système de fichiers (FAT, FAT12, FAT16).
\begin{verbatim}
	00000020 | 00 00 00 00 00 00 29 a5 | 0e ae eb 20 20 20 20 20   |......)....     | 
	00000030 | 20 20 20 20 20 20 46 41 | 54 31 32 20 20 20 0e 1f   |      FAT12   ..| 
\end{verbatim}
\newpage

\subsection{FAT32}
\begin{verbatim}
00000000 | eb 58 90 6d 6b 64 6f 73 | 66 73 00 00 02 01 20 00   |.X.mkdosfs.... .|
00000010 | 02 00 00 00 00 f8 00 00 | 20 00 40 00 00 00 00 00   |........ .@.....|
00000020 | 00 00 02 00 f1 03 00 00 | 00 00 00 00 02 00 00 00   |................|
00000030 | 01 00 06 00 00 00 00 00 | 00 00 00 00 00 00 00 00   |................|
00000040 | 00 00 29 be 55 74 03 20 | 20 20 20 20 20 20 20 20   |..).Ut.         |
00000050 | 20 20 46 41 54 33 32 20 | 20 20 0e 1f be 77 7c ac   |  FAT32   ...w|.|
00000060 | 22 c0 74 0b 56 b4 0e bb | 07 00 cd 10 5e eb f0 32   |".t.V.......^..2|
00000070 | e4 cd 16 cd 19 eb fe 54 | 68 69 73 20 69 73 20 6e   |.......This is n|
00000080 | 6f 74 20 61 20 62 6f 6f | 74 61 62 6c 65 20 64 69   |ot a bootable di|
00000090 | 73 6b 2e 20 20 50 6c 65 | 61 73 65 20 69 6e 73 65   |sk.  Please inse|
000000a0 | 72 74 20 61 20 62 6f 6f | 74 61 62 6c 65 20 66 6c   |rt a bootable fl|
000000b0 | 6f 70 70 79 20 61 6e 64 | 0d 0a 70 72 65 73 73 20   |oppy and..press |
000000c0 | 61 6e 79 20 6b 65 79 20 | 74 6f 20 74 72 79 20 61   |any key to try a|
000000d0 | 67 61 69 6e 20 2e 2e 2e | 20 0d 0a 00 00 00 00 00   |gain ... .......|
000000e0 | 00 00 00 00 00 00 00 00 | 00 00 00 00 00 00 00 00   |................|
*
000001f0 | 00 00 00 00 00 00 00 00 | 00 00 00 00 00 00 55 aa   |..............U.|

\end{verbatim}

16 premiers octets : \\
3 Octets : eb 58 90 : Saut vers un programme qui va charger le système d'exploitation \\
8 Octets : mkdosfs : Nom du programme qui a formaté le disque\\
2 Octets : 00 02 : Nombre d'octets par secteur (512, 1 024, 2 048 ou 4 096).\\
1 Octet : 01 : Nombre de secteurs par cluster (1, 2, 4, 8, 16, 32, 64 ou 128).\\
2 Octets : 02 00 : Nombre de secteurs réservés en comptant le secteur de boot\\
(32 par défaut pour FAT32, 1 par défaut pour FAT12/16).
\begin{verbatim}
00000000  eb 58 90 6d 6b 64 6f 73  66 73 00 00 02 01 20 00  |.X.mkdosfs.... .|
\end{verbatim}

16 seconds octets :\\
1 Octet : 02 : Nombre de FATs sur le disque (2 par défaut)\\
2 Octets : 00 00 : Taille du répertoire racine en nombre d'entrées.\\
2 Octets : 00 00 : Nombre total de secteurs 16-bit.\\
1 Octet : f8 : Type de disque (0xF8 pour les disques durs, 0xF0 pour les disquettes).\\
2 Octets : 00 00 : Taille d'une FAT en secteurs.\\
2 Octets : 20 00 : Nombre de secteurs par piste.\\
2 Octets : 40 00 : Nombre de têtes.\\
4 Octets : 00 00 00 00 : Secteurs cachés (0 par défaut si le disque n'est pas partitionné).
\begin{verbatim}
00000010  02 00 00 00 00 f8 00 00  20 00 40 00 00 00 00 00  |........ .@.....|
\end{verbatim}
\newpage

Spécifiques aux FAT32 : \\
4 Octets : 00 00 02 00 : Nombre total de secteurs 32-bit.\\
4 Octets : f1 03 00 00 : Taille d'une FAT en secteurs (remplace l'équivalent cité au-dessus)\\
2 Octets : 00 00 : Attributs du disque.\\
1 Octet : 00 : Version majeure du système de fichiers.\\
1 Octet : 00 : Version mineure du système de fichiers.\\
4 Octets : 02 00 00 00 : Numéro du premier cluster du répertoire racine.
\begin{verbatim}
00000020 | 00 00 02 00 f1 03 00 00 | 00 00 00 00 02 00 00 00   |................|
\end{verbatim}
2 Octets : 01 00 : Informations supplémentaires sur le système de fichiers (1 par défaut).\\
2 Octets : 06 00 : Numéro de secteur de la copie du secteur de boot.\\
12 Octets : 00 00 00 00 00 00 00 00 00 00 00 00 : Réservé pour des ajouts ultérieurs
\begin{verbatim}
00000030 | 01 00 06 00 00 00 00 00 | 00 00 00 00 00 00 00 00   |................|
\end{verbatim}
1 Octet : 00 : Identifiant du disque \\(à partir de 0x00 pour les disques amovibles et à partir de 0x80 pour les disques fixes).\\
1 Octet : 00 : Réservé pour usage ultérieur.\\
1 Octet : 29 : Signature (0x29 par défaut).\\
4 Octets :  be 55 74 03: Numéro de série du disque.\\
11 Octets : 20 20 20 20 20 20 20 20 20 20 20 : Nom du disque sur 11 caractères.\\
8 Octets : 46 41 54 33 32 20 20 20 : Type de système de fichiers (FAT32).
\begin{verbatim}
00000040 | 00 00 29 be 55 74 03 20 | 20 20 20 20 20 20 20 20   |..).Ut.         |
00000050 | 20 20 46 41 54 33 32 20 | 20 20 0e 1f be 77 7c ac   |  FAT32   ...w|.|
\end{verbatim}
\newpage

\section{FAT 12}
La File Allocation Table va indiquer quels sont les clusters utilisés par les fichiers, et de quel façon le fichier a été découpé s'il est plus gros qu'un cluster.
Comme indiqué dans le nom, la FAT12 utilise des entrées de 12 bits. \\
Les valeurs prisent dans la FAT peuvent prendre différentes valeurs, indiquant : \\
0x000 : Le cluster est vide.\\
0x001 : Le cluster est réservé.\\
0x002 - 0xFEF : Le cluster contient un fichier, continue dans le cluster indiqué par ce numéros.\\
0xFF0 - 0xFF6 : valeurs réservées.\\
0xFF7 : Mauvais cluster.\\
0xFF8 - 0xFFF : Le cluster contient la fin d'un fichier.\\
\section{FAT 16}
La File Allocation Table va indiquer quels sont les clusters utilisés par les fichiers, et de quel façon le fichier a été découpé s'il est plus gros qu'un cluster.
Comme indiqué dans le nom, la FAT16 utilise des entrées de 16 bits. \\
Les valeurs prisent dans la FAT peuvent prendre différentes valeurs, indiquant : \\
0x0000 : Le cluster est vide.\\
0x0001 : Le cluster est réservé.\\
0x0002 - 0xFFEF : Le cluster contient un fichier, continue dans le cluster indiqué par ce numéros.\\
0xFFF0 - 0xFFF6 : valeurs réservées.\\
0xFFF7 : Mauvais cluster.\\
0xFFF8 - 0xFFFF : Le cluster contient la fin d'un fichier.\\

\section{FAT 32}
La File Allocation Table va indiquer quels sont les clusters utilisés par les fichiers, et de quel façon le fichier a été découpé s'il est plus gros qu'un cluster.
Comme indiqué dans le nom, la FAT32 utilise des entrées de 32 bits.\\
Les valeurs prisent dans la FAT peuvent prendre différentes valeurs, indiquant : \\
0x00000 : Le cluster est vide.\\
0x00001 : Le cluster est réservé.\\
0x00002 - 0xFFFEF : Le cluster contient un fichier, continue dans le cluster indiqué par ce numéros.\\
0xFFFF0 - 0xFFFF6 : valeurs réservées.\\
0xFFFF7 : Mauvais cluster.\\
0xFFFF8 - 0xFFFFF : Le cluster contient la fin d'un fichier.\\

\section{Root Directory}
Après les FAT viendra la liste d'entrée du répertoire principal, chaque entrées étant réparties sur 32 bits de la façon suivante : \\
8 Octets : 4d 4f 56 49 45 2d 7e 31 : Nom du fichier\\
3 Octets : 33 47 50 : Extension du fichier\\
1 Octet : 20 : Attributs du fichier\\
Fonctionne par masque, un fichier peut avoir 0x03 = 0000 0011 = 0x01 + 0x02\\
0x01	Lecture seule\\
0x02	Fichier caché\\
0x04	Fichier système\\
0x08	Nom du volume\\
0x10	Sous-répertoire\\
0x20	Archive\\
0x40	Device (utilisé en interne, jamais sur disque)\\
0x80	Inutilisé\\
0x0F est utilisé pour désigner un nom de fichier long.\\
1 Octet : 00 : Réservé, utilisé par NT\\
1 Octet : 00 : Heure de création : par unité de 10ms (de 0 à 199)\\
2 Octets : e2 72 : Heure de création
\begin{verbatim}
000026c0 | 4d 4f 56 49 45 2d 7e 31 | 33 47 50 20 00 00 e2 72    |MOVIE-~13GP ...r|
000026d0 | b6 42 b6 42 00 00 d7 72 | b6 42 05 00 76 d3 00 00    |.B.B...r.B..v...|
\end{verbatim} 
2 Octets : b6 42 : Date de création\\
2 Octets : b6 42 : Date du dernier accès\\
2 Octets : 00 00 : \\
FAT12/16 : Index EA, utilisé par OS/2 et NT\\
FAT32 : premier cluster du fichier, octet de poids faible \\
2 Octets : d7 72 : Heure de dernière modification\\
2 Octets : b6 42 : Date de dernière modification \\
2 Octets : 05 00 : \\
FAT12/16 : Numéros du premier cluster du fichier\\
FAT32 : Cluster du fichier, octets de poids fort \\
4 Octets : 76 d3 00 00  : Taille du fichier \\
\newpage

\section{Cas des noms de fichier long}
Les entrées à long nom sont écrites dans la mémoires de la façon suivante : \\
Ces entrées sont marquées des attribut nom de volume, système, caché, lecture seule (valeur 0x0F) a l'ofset numéros 12.\\
1 Octet : 42, 01 : Ordre dans la séquence. Si contient le bit de masque 0x40, indique qu'il s'agit de la dernière entrée de ce nom. \\
10 Octets : 2e 00 64 00 6f 00 63 00 00 00 : Nom du fichier, 5 caractères\\
1 Octet : Attribut spécial, 0x0F\\
1 Octet : réservé, utilisé par NT\\
1 Octet : Heure de création, par unité de 10ms\\
12 Octets : Nom du fichier, 6 carractères\\
2 Octets : par soucis de compatibilité, mis à 0 \\
4 Octets : Nom du fichier, 2 carractères\\
\begin{verbatim}
0001f200 | 42 2e 00 64 00 6f 00 63 | 00 00 00 0f 00 c0 ff ff    |B..d.o.c........|
0001f210 | ff ff ff ff ff ff ff ff | ff ff 00 00 ff ff ff ff    |................|

0001f220 | 01 32 00 30 00 30 00 33 | 00 5f 00 0f 00 c0 64 00    |.2.0.0.3._....d.|
0001f230 | 6f 00 63 00 75 00 6d 00 | 65 00 00 00 6e 00 74 00    |o.c.u.m.e...n.t.|

0001f240 | 32 30 30 33 5f 44 7e 31 | 44 4f 43 20 00 00 39 6b    |2003_D~1DOC ..9k|
0001f250 | 69 32 69 32 00 00 39 6b | 69 32 02 00 00 4e 00 00    |i2i2..9ki2...N..|
\end{verbatim}
\chapter{NTS}

3 Octets :  Jump instruction\\
8 Octets : OEM ID\\
25 Octets : BPB\\
48 Octets : Extended BPB\\
426 Octets : : Bootstrap code\\
2 Octets : End of sector marker\\


\chapter{EXT}

\section{EXT 2}

\section{EXT 3}

\section{EXT 4}

\appendix
\chapter{Calcul de Date / Heure}
Le calcul d'une date s'effectue sur 2 octets de la façon suivante : \\
Les 2 octets sont reformés en 32 bits tels que : \\
Bits 15-9 : Année - 1980 \\
Bits 18-5 : Mois \\
Bits 0-4 : Jour \\
L'heure fonctionne de façon très similaire sur 32 bits : \\
Bits 15-11 : Heures \\
Bits 10-5 : Minutes \\
Bits 0-4 : Secondes/2 \\
\end{document}
